% -*- coding: utf-8 -*-
\documentclass{llncs}

\usepackage[utf8]{inputenc}
\usepackage[T1]{fontenc}

\usepackage{xskak}
\usepackage{chessboard}

\usepackage{amsmath,amssymb,cite,graphicx}
\usepackage{subfigure}
%\usepackage{subcaption}
%\usepackage{subfig}


%\usepackage[left=1cm,right=1cm,top=2cm,bottom=2cm]{geometry}
%\usepackage{graphicx,booktabs}
%\pagestyle{empty}
%\setlength{\parindent}{0pt}



\author{Anastasia Afanaseva \and Sergey Afonin}
\institute{Moscow State University, Russian Federation\\
}%\email{serg@msu.ru}, \email{serg@msu.ru}}
\title{Computer Chess in a Human Style: is it Feasible?}

%\date{}

\begin{document}
\maketitle

\begin{abstract}
  For extended period of time it was widely accepted that
  computers can not play chess in the human style. Top level chess
  programs follow the game tree exploration approach analyzing
  huge number of positions and possible moves. To our best
  knowledge no successful realization of a program following an
  alternative approach, such as the one introduced by M.M.~Botvinnik,
  was ever conducted.
  % 
  In this paper we analyze possibility of such implementation. Number
  of optimization problems are identified and corresponding algorithms
  are proposed.

  % consider approach to finding chess plan, introduced
  % by M.M.~Botvinnik, aimed to model cognitive processes of a human
  % player.
  \keywords{Computer chess, chunking, optimization problems.}
\end{abstract}

\section{Introduction}
For more than over decade computer chess programs outperform top-level
human players. This advance was determined by the significant
improvements of computer hardware. Increased computational power and
capacity of fast internal memory make it possible to run brute-force
algorithms with appropriate pruning strategies in reasonable running
time. Tablebases for 7-pieces endgames, computed at Moscow State
University on <<Lomonosov>> supercomputer in 2012, contains funny
examples such as proved mate in 545 moves. Nevertheless, brute-force
algorithms differ from the humans' cognitive processes and one of the
most important goal of research on computer chess, i.e. modeling of
humans behavior during complex problems solving, was not achieved in
general. Such modeling require deep knowledge from neurophysiology,
psychology, computer science and many other areas. Continuing research
on such topics, e.g.~\cite{reingold:2005:perception,CAMPITELLI2007}, indicates
importance of the problem.

In this paper we consider quite dated approach, frequently referenced to
as <<the M.M.~Botvinnik's method>>, aimed to model cognitive processes
of a human chess player. This choice is personally motivated because
one of the authors was a member of the last Botvinnik's team.  The key
notion in this approach is a \emph{chain}~--- a piece's trajectory
with all relevant supporting and defending moves. Let us illustrate it
by example. Suppose, that White is going to move a Pawn from a5 to a8
by a sequence of moves a5--a6--a7--a8Q. If some square, say a7, is
occupied by a White's piece, then the original plan requires
supporting moves, say Na7-c8. If this move is not possible (or not
acceptable by some reason), extra supporting moves might be required
to make it happens. Similarly, Black is trying to defend either by
blocking the trajectory or by taking some squares under control, so
Bg6-e4-h8 might also be in that chain. In turn, White might attack e4
in order to invalidate Black's defense, and so on.

In some sense a chain represents plan of playing. Starting with an
idea one can find all obstacles preventing it's realization, and
consider possible ways of elimination of such obstacles. Many
optimization problems fit this terminology and the general approach
might be useful to solve over problems. Attempts for such
generalization were made, e.g.~\cite{Botvinnik:1984,Botvinnik1970,Stilman:2000:LGS},
but in our opinion any generalization is premature before the
implementation of a master level chess program. Generalization
introduces extra level of complexity because one should explain rules,
goals and heuristics in terms of a general framework which is not
always possible.

Chain construction is a complex problem. One piece might have several
chains in general. When all chains (for both players) are constructed,
consistent representation of the current position should be
constructed by merging the chains. At this stage one may discover that
a move in one chain is also a move in another, e.g. fork, or using a
piece for supporting a chain is not possible because this piece
required in a more valuable chain. Having a consistent representation
one can determine \emph{candidate moves} and run their validation by
moving pieces on a virtual board.
%
In contrast to game tree exploration procedure dominating in the
modern chess programs, chains are constructed using mostly static
position, the set of candidates moves are expected to be very small,
and the total number of positions visited on validation stage shall
not exceed one hundred positions. According to M.M.~Botvinnik the
described procedure reflects human player's behavior.

The possibility of successful development of human-like computer chess
program in general, and the one following Botvinnik's method in
particular, was widely criticized, e.g.~\cite{Berliner:1993:Playing}.
According to our best knowledge such programs were never developed and
this is certainly so for Botvinnik's method. In this work we are
trying to formulate optimization problems and decision questions that
should be answered in order to implement the general idea. The main
question is: Can this approach, stated in the 70s, benefits from
several orders of magnitude increase of memory capacity and CPU
performance, or it is broken by itself?
%

%Chain and position representation construction are hard combinatorial
%optimization problems.



\section{Brief Description of the Approach}\label{sec:general}
In this section we briefly describe the approach followed by this
paper. It is mostly based on Botvinnik's ideas on decomposition of
position into smaller parts, similarly to chunking theory of de Groot,
Simon and Chase. It was originally expected that this approach is not
only the basis for a chess program, but a tool for solving a wider
class of optimization problems as well. For example, this approach was
applied to some economics problem, leading to a well known economics'
model. Nevertheless, we will use chess terminology in the reset of the
paper.

As we have mentioned early, the key notion of the approach is a chain.
A chain is associated to a piece and determined by piece's trajectory
form its current position on the board to a desired square. This
trajectory is called \emph{subchain-0}. If a square on the trajectory
in not <<passable>> (either blocked by another piece, or exchange
value on the square is not acceptable) then supporting subchain-1 is
needed. Subchains-2 are used in support for subchains-1, and so forth.


At the very top level the overall algorithm can be decomposed into
following stages:
\begin{itemize}
\item for every piece find appropriate \emph{goals} that should be
  achieved (e.g., a square to be reached, or opponent's piece to be captured);
\item for every goal construct \emph{subchain-0}~--- a \emph{trajectory} that
  might be used by the piece for achieving its goal;
\item validate that subchain-0 is realizable, or extend subchain-0 by
  supporting subchains-1 (e.g. take away other pieces from the
  subchain-0's trajectory);
\item construct all the necessary supporting subchains of higher
  orders (subchain-2 is a supporting subchain for a subchain-1);
\item merge all the chains into consistent representation of the current position;
\item find candidate moves;
\item validate candidate moves by tree searching.
\end{itemize}
Such scheme assumes that chains are constructed independently and then
merged together, so the problem divided into two parts: chain
construction, and chains merging. Chains are used for position and
piece evaluation. Roughly speaking, a valuable piece participates in
large number of valuable chains, and there are many valuable attacking
chains in a good position.

Many aspects should be taken into consideration. It is difficult to
find a real-life example that can be described using one chain. In the
rest of this section we describe questions that should be addressed by
an algorithm for construction position representation.

\paragraph{Large number of chains.} In order to be able to find a
reasonable plan of playing, subchains-0 should be sufficiently
long. If a goal is specified by the target square, the simplest form
of a goal, then the set of subchain-0 candidates consists of all
trajectories of the piece on the empty board, starting from its
current position and ending on the target square. The \emph{horizon},
counted in plies, may vary depending on piece's kind or stage of the
game. Nevertheless, the total number of chains, i.e. trajectories
passed some filtering condition, could be large.

\paragraph{High variability of chains.} Every chain contains the only
subchain-0. If subchains of higher orders are used in that chain, then
one can expect high ambiguity between subchains addressing the same
problem. For example, if the attacker need to take some square on the
subchain's-0 trajectory under control, then it could be done in many
ways, in general. Number of variations growths rapidly with the
maximum subchain's order. In case of independent construction of
chains all possibilities should be preserved or enumerated later when
additional information from other chains become available.

\paragraph{Timing.} Defending subchains are time-constrained. It is
not only required to defend against opponent's subchain, but do it in
time. The more plies required for the attacker to reach a square, the
longer defending subchains are possible. One can easily compute a raw
estimation of number plies required to reach given square on the
subchain's trajectory, but this value might change depending on
modification of subchains associated with preceding squares.

\paragraph{Functional dependencies between chains.} Clearly, one piece
may participate in many chains. A fork is the most natural example. If
such overlapping accrues then a move in one chain is also a move in
another one. As a result timing constraints in one of overlapping
chains change.

\paragraph{Dynamic nature of entities' values.} In order to compute the value of a chain,
or to decide whether or not a square is <<passable>> it is required to
compute the exchange value that depends on values of affected
pieces. Piece value is a function of its chains. Once pieces get their
new values some chains might change their structure~--- a piece
intended to participate in the chain could be too valuable for that.

% Kotov-Botvinnik, 1955
% Kc3,Bc5,Pe3,Pf4,Ph4,kf3,be6,pb3,pd5,pg6,ph5


\begin{figure}[t]
  \centering
  \subfigure[Kotov~--~Botvinnik, 1955. 59. \ldots?]{%
    \label{pos:Kotov-Botvinnik}%
    \scalebox{0.8}{%
      \setchessboard{showmover=true}%
      \chessboard[setpieces={Kc3,Bc5,Pe3,Pf4,Ph4,kf3,be6,pb3,pd5,pg6,ph5},%
      arrow=stealth,%
      linewidth=.25ex,%
      padding=1ex,%
      color=red!75!white,%
      pgfstyle=straightmove,%
      shortenstart=1ex,%
      showmover=false,%
      markmoves={h5-h1},%
      pgfstyle=straightmove,%
      color=green!75!white,%
      padding=10ex,%
      shortenstart=0ex,%
      shortenend=1ex%
      %, markmoves={f3-g3,g3-h4}%
      ]%
    }%
  }
  \subfigure[Ivanchuk~--~Yusupov, 1991. 23. \ldots?]{
    \label{pos:Ivanchuk-Yusupov}
    \scalebox{0.8}{%
      \setchessboard{showmover=true}
      \chessboard[addfen={r3r1k1/p4pb1/2NB2np/3N1b1q/2PP1pnP/1Q2p1P1/P3P1B1/R2R2K1 b - h3 0 23},
      pgfstyle=straightmove,
      arrow=stealth,
      linewidth=.25ex,
      padding=1ex,
      color=red!75!white,
      pgfstyle=straightmove,
      shortenstart=1ex,
      showmover=false]
    }
  }
\end{figure}

% Ivanchuk-Yousupov, 1991
% r3r1k1/p4pb1/2NB2np/3N1b1q/2PP1pnP/1Q2p1P1/P3P1B1/R2R2K1 b - h3 0 23
% Kasparov-Ribli, 1989
% 5rk1/5ppp/p1Q1p3/1R6/q7/4b1P1/P2RPP1P/6K1 w - -


Let us consider position on Fig.~\ref{pos:Kotov-Botvinnik}. This
position was always among Botvinnik's tests. The winning plan for
Black is h5-h1. Pawn can not move on h4, so a supporting subchain-1
Kf3-g3-h4 or Kf3-g4-h4 required. (Actually, Kf3-g4-h4 is not suffice,
as mentioned below). Immediate attack on h4 is not successful because
of White's defense Bc5-e7 (subchain-2). White's defense can be
eliminated by g6-g5, if played before Bc5-e7. Complete human analysis
is as follows.

%\longmoves 
\xskakset{style=UF}
\newchessgame[
moveid=1b, %print,
showmover,
result=0--1,
mover=b, % has no effect
castling=Q,
enpassant=a3,
setwhite={pa4,pc7,pg5,ph2,pe4,ke1,ra1,bg4},
addblack={pb4,ph7,pd5,ke8,rh8,bh3,nf6,ne5}]
%
\mainline{1... g5!! 2. fxg5 d4 3. exd4}
  { [\variation[invar]{ 3. Bxd4 Kg3 4. g6 Kxh4 5. Kd2 Kh3!! 6. Bf6 h4 7. Ke2 Kg2 }] }
\mainline{ 3... Kg3 } 
  {[\variation[invar]{ 3... Kg4 4. d5 Bxd5 5. Bf2}]}
\mainline{4. Ba3 }
  {[{\variation[invar]{4. g6 Kxh4 5. g4 Kg4}}, or \variation[invar]{4. Be7 Kxh4 5. g6+ Kg4}] }
\mainline{ 4... Kxh4 5. Kd3 Kxg5 6. Ke4 h4 7. Kf3 Bd5+ }
\mbox{\xskakgetgame{result}}
%

Note, that {\bf 2... d4} considered as an obvious move: Black's Bishop
should a) protects Pawn on b3 in order to keep the threat of
b3-b2-b1Q, and b) controls g8 square, making White's plan g5-g6-g7-g8
impossible. From the computer's perspective this move is not
self-evident at all. One should discover that Pawn on b3 is a crucial
piece: White's King becomes overloaded in some variations, trying to
defend against h5-h1 and b3-b1 simultaneously.

This example demonstrates that even a simple position with less than a
dozen pieces could provides plenty of variations with complex
correlations between different chains (plans). In more complex
positions the number of possibilities could be enormous. For example,
in the position depicted on Fig.~\ref{pos:Ivanchuk-Yusupov}, there are
about $120\,000$ subchain-0 candidates, so importance of performance
issues can not be underestimated.

\section{Chain Construction}
We consider chain mostly as a data structure for storing possible
variations, trying to keep construction procedure as simple as
possible. Chain is not required to be absolutely precise. All
non-trivial decisions and improvements are delayed for later
stages. Nevertheless, chain construction is far from trivial. A chain
should represents best variation using given values for pieces. In
this section we discuss issues of this procedure.

The chain is a recursive structure containing trajectory of the main
piece (subchain-0) and, possibly, a number of subchains associated
with squares on this trajectory. Given a subchain-0 trajectory,
e.g. Qh5-h2-f3 on Fig.~\ref{pos:Ivanchuk-Yusupov}, we first build the
extended trajectory by including all the intermediate squares (h4, h3,
g2). Squares of the extended trajectory divided into two classes:
stop-squares, where the piece intended to stop, and internal
squares. We call a stop-square \emph{passable} if it is not occupied
by another piece of the same color and the exchange value on that
square is positive. An internal-field is passable if it is free, and,
in case of castling, not under control.

For each non-passable field supporting subchains should be
constructed. If the square is blocked by a piece of the same color as
the chain's piece, then blocking piece should escape somewhere
first. In case of unfavorable exchange value (let us recall that we
assume that pieces' values depend on chains they are involved)
following possibilities exist:
\begin{itemize}
\item take the square under control by a piece that can improve exchange value;
\item exclude opponent's piece from the exchange by pinning it, or blocking its trajectory;
\item capture opponent's piece.
\end{itemize}
It is possible that only a combination of actions give positive
exchange value on the square. So it is reasonable to consider bunch of
subchains~--- a set of supporting chains that make sense only if
realized together. Clearly, a bunch may be a singleton set. As chains
are constructed independently from each other it is not possible to
decide at this point which bunch is the best. A chain contains locally
optimal choice, i.e. the bunch that yields best result in this
chain. Other possibilities are recorded, probably in less details. If
the selected bunch will be rejected on a later stage as a result of
chains merging then the chain should be updated and previously
recorder subchain will be analyzed in more details, i.e. subchains of
higher orders will be constructed.

Subchains are constructed for both colors. There is a crucial
difference between supporting and defending subchains. Defending
subchains are time-limited. For every square on subchain-0 trajectory
one can estimate number of moves required for main piece to reach that
square, including moves in subchains associated with preceding
squares. Defending subchain should not require more time. Non-trivial
timing constraints frequently appear in the endgame.  for example, It
should be mentioned time constraints are fuzzy. <<Slow>> defending
subchains, that require more plies than available, should also be
included into the chain: playing some moves in a slow subchain might
be for free due to overlapping of that moves with more valuable
chains. The most obvious example of such overlapping is a check. If
slow subchains will be ruled out at chain construction stage then they
will be completely eliminated from consideration.

Every non-passable square introduces extra uncertainty and complexity
to the chain, especially if there are several non-passable fields in
one chain. Same pieces might participate in supporting subchains for
different squares. For example, if a chain contains two non-passable
squares, some piece might leave its position in order to support main
chain on the first non-passable square. This movement should be
reflected while searching for support subchains for the second
square. Similar dependencies appear between subchains of different
levels. For example, first move in subchain-0 might also supports
subchain-1 associated with another square of subchain-0.

When a chain is constructed a correct order of moves should be
determined.  In position on Fig.~\ref{pos:Kotov-Botvinnik} square h4
is non-passable and there are three possible supporting subchains,
namely Kf3-g3-h4, Kf3-g4-h4, and g6-g5. Black can play any of the
three moves, while only one wins.

Summing up:
\begin{itemize}
\item chains are constructed independently, regardless of other threats presented in current position;
\item each chain contains selected set of subchains that is optimal if this chain considered isolated;
\item for each bunch of supporting and defending subchains a chain contains all possible variations, i.e. other chains;
\item for each chain possible sequence of moves, or a set of equivalent sequences, is defined.
\end{itemize}


\section{Consistent Position Representation}
Chains are more or less <<local>>. Once all the chains are constructed
one can estimate dynamic values for pieces. Change of pieces' values
affects the result of chain construction procedure, which in turn
influences piece values. It is unlikely that simple iterative
procedure would converge to a certain if pieces' values will change
drastically.

Regardless of pieces' values convergence there are many cases where
chains should be altered. The following types of chains interaction
are among most frequent cases.
\begin{itemize}
\item Fork, or subchains overlapping. One move reveals two or more
  threats simultaneously. As we have mentioned early, a fork might
  appear between subchains on any level. It is possible that one of
  the overlapping subchain does not belong to it's chain main
  (selected by chain construction procedure) variation.
\item Attraction. Forcing a piece to an unfavorable square.
\item Piece overloading, zugzwang. For every chain, while considered
  independently, the goal is unreachable, but there is at least one
  piece that appears in both chains.
\end{itemize}

\section{Conclusion and Future Work}
In this paper Botvinnik's approach to chess playing was considered. We
identified some questions that have to be answered in order to
implement one of its main part~-- chain construction procedure. It is
absolutely unclear at the moment whether or not consistent position
representation can be constructed from individual chain in reasonable
amount of time.

Direction of future work includes implementation of the described
procedure and validation of the approach by means of eye
tracking. Modern wearable glasses are accurate enough to track eye
movements over a chess board without inconveniences to the
subject. Provided that a working implementation for chain construction
procedure exists (which is not enough for chess-playing program) it is
possible to compare actual eye movement with chains computed by the
program.

The method considered in this paper falls into class of expert
system. As a very distant perspective, in case of successful
implementation of a working program, one may consider application of
machine learning
techniques~\cite{Fogel:2004:self-learning,Wirth:2015:on-learning} for
parameters estimation.

\bibliographystyle{acm}
{\small
\bibliography{chess}}

\end{document}
