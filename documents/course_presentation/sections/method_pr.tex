%\subsection{Подход М.М.Ботвинника}
\begin{frame}{Подход М.М.Ботвинника}
Ботвинник предлагал анализировать расположение фигур на статической доске. Для этого необходимо: 
\begin{itemize}
\item определить локальные цели для данной позиции;
\item наметить план достижения цели;
\item проверить осуществимость плана;
\item объединить планы в единое представлене о позиции;
\item определить оптимальную последовательность действий.
\end{itemize}
Данный подход рассматривался, как средство решения широкого класса комбинаторно-оптимизационных задач.
\end{frame}

\subsection{Основные понятия}
\begin{frame}{Основные понятия}
\begin{description}
\item[Траектория] последовательность полей движения фигуры на пустой доске
\item[Горизонт] максимальная длина рассматриваемых траекторий
\item[Цель] фигуры противника или поля
\item[Подцепочка] действия, направленные на достижение цели
\item[Цепочка] совокупность действий, направленных на достижение цели
\item[Вилочность] совпадение частей цепочек
\item[Оценка фигуры] числовая характеристика <<полезности>> фигуры
\item[Время успевания] допустимое число ходов для защиты от атаки противника
\end{description}
\end{frame}

\begin{frame}{Основные понятия: пример цепочки}
\begin{columns}
\column{0.5\textwidth}
\begin{figure}[t]
  \centering
  \subfigure[1.]{%
    \scalebox{0.8}{%
      \setchessboard{showmover=true}%
      \chessboard[setpieces={Pa5, Na7, bh3},%
      arrow=stealth,%
      linewidth=.25ex,%
      padding=1ex,%
      color=red!55!white,%
      pgfstyle=straightmove,%
      shortenstart=1ex,%
      showmover=false,%
      markmoves={a5-a8},%
      padding=10ex,%
      shortenend=1ex%
      %, markmoves={f3-g3,g3-h4}%
      ]%
    }%
  }
\end{figure}
\column{0.5\textwidth}
\textbf{Цель} - достижение пешкой поля a8 \\
\textbf{Траектория} - a5-a6-a7-a8. \\
\pause
\textbf{Подцепочка-1} - освобождение траектории a7-b5, a7-c6, a7-c8. \\
\pause
\textbf{Подцепочка-1} - защита противника h3-g2-(a8), h3-f1-(a6) \\
\pause
\textbf{Подцепочка-2} - поддержка a7-b5-c7-(a8, a6), a7-c8-b6-(a8)
\end{columns}
\end{frame}


\subsection{Примеры и обобщения}
\begin{frame}{Примеры на модельных задачах} %без королей
Позиция, \\
анализ, \\ 
пример цепи (цель). 
\end{frame}

\begin{frame}{Обобщение}
Лингвистическая геометрия \\
Экономика
\end{frame}

\begin{frame}{Проблемы} % +диаграммы на каждый пункт с примерами 
Динамический горизонт событий \\
Динамическая оценка фигур и цепей \\
Связь между цепочками \\
Траектории с подвижной целью \\
Оптимальность цепи
%НЕТ ФОРМАЛИЗАЦИИ!! \\
\end{frame}
