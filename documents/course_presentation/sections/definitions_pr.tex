%\subsection{Как думает человек}
\begin{frame}{Психология принятия решений} 
Игра в шахматы является традиционным предметом изучения психологии принятия решений. Первые работы опубликованы в 1894 году. 

\bigskip
Большой вклад внёсли работы А. де Гроота (1946), Саймона (1982) и другие. Проводились исследования по двум основным направлениям:
\begin{itemize}
\item Восприятие (запоминание позиции)
\item Принятие решения при выборе хода
\end{itemize} 
\end{frame}

\begin{frame}{Восприятие}
\textbf{Эксперимент} 
\begin{itemize}
\item На время $t_1 = 5$с предъявлялась позиция. Через время $t_2 = 30$с предлагалось её восстановить. Анализровался порядок и число правильно восстановленных фигур. 
\end{itemize}
\textbf{Результаты}
\begin{enumerate}
\item На реальных позициях есть различие между экспертами и новичками.
\item Для случайных позиций различия нет.
\item Фигуры рассматриваются логически связанными группами.
\item Некоторые позиции восстанавливаются по ассоциации с другими партиями.
\end{enumerate}
\end{frame}


\begin{frame}{Принятие решения}
\textbf{Эксперимент}
\begin{itemize}
\item Экспертам предъявлялась позиция из реальной партии. Предлагалось выбрать наилучший ход. Производить рассуждения требовалось вслух.   
\end{itemize}
\textbf{Результаты}
\begin{enumerate}
\item Рассматривается небольшое число ходов-кандидатов.
\item Небольшая глубина анализа (до 5 ходов) вне зависимости от мастерства.
\item Дерево анализа содержит $20-80$ узлов. 
\item Частые возвращения назад: анализ хода или эпизода часто повторяется.  
\end{enumerate}
\end{frame}

\begin{frame}{Теория прогрессивного углубления}
По результатам проведенных исследований де Гроотом были выделены основные фазы мышления шахматиста\footnotemark{}
\begin{itemize}
\item ориентировка (выделение группы ключевых фигур);
\item разведка (пробы нескольких ходов);
\item обработка (систематический глубокий просчёт вариантов);
\item доказательство (проверка надёжности результата).
\end{itemize}
\footnotetext{de Groot A.D., 1946, 2008. Thought and Choice in Chess,  Amsterdam University Press // Amsterdam Academic Archive.}
\end{frame}